\documentclass[a4paper,10pt]{article}
\usepackage[utf8x]{inputenc}
\usepackage{listings}
\usepackage{a4wide}
\usepackage{tabularx}
\usepackage{booktabs}

\lstloadlanguages{Bash}
\lstset{basicstyle=\small\ttfamily, escapeinside={(*@}{@*)},
showstringspaces=false, breaklines=true, breakatwhitespace=true,
tabsize=4, language=Bash, numberfirstline=true,
commentstyle=\itshape\color{CommentGreen},
stringstyle=\color{DataTypeBlue}}
\newcommand{\ilcode}[1]{\lstinline|#1|}

\title{RTC:HokuyoAist User Guide}
\author{Geoffrey Biggs\\
geoffrey.biggs@aist.go.jp}

\begin{document}
\maketitle

\section{Introduction}
\label{sec:intro}

RTC:HokuyoAist is an RT Component for the OpenRTM-aist middleware. It provides
a driver for Hokuyo laser range sensors. It wraps the hokuyo\_aist library from
Gearbox\footnote{http://gearbox.sourceforge.net/}. It functions with all
current models of laser scanner, including the URG-04LX (Classic-URG), UHG-08LX
(Hi-URG), UTM-30LX (Top-URG) and UXM-30LX/UXM-30LX-E (Tough-URG).

The laser remains off until the component is activated. The laser is then
turned on, unless the component is configured to be in pull mode. On start-up,
the laser's internal clock is calibrated to the computer's clock, including the
communications delay. Time stamps in the range data are calculated using this
calibrated time, so no calibration is necessary by users of the data.

This software is developed at the National Institute of Advanced Industrial
Science and Technology. Approval number H22PRO-1194. The development was
financially supported by the New Energy and Industrial Technology Development
Organisation Project for Strategic Development of Advanced Robotics Elemental
Technologies.  This software is licensed under the Eclipse Public License -v
1.0 (EPL). See LICENSE.TXT.

\section{Requirements}
\label{sec:requirements}

RTC:HokuyoAist uses the Gearbox ``hokuyo\_aist''
library\footnote{http://gearbox.sourceforge.net/group\_\_gbx\_\_library\_\_hokuyo\_\_aist.html}
for its functionality. At least version 2.0.0 is required\footnote{At the time
  of writing, version 2.0.0 has not been released. You must download it from
  the SVN repository.}. A compiled copy of this library is included in the
installer for users of Windows.

RTC:HokuyoAist requires the C++ version of OpenRTM-aist-1.0.0.

RTC:HokuyoAist uses the CMake build system\footnote{http://www.cmake.org/}. You
will need at least version 2.6 to be able to build the component.

RTC:HokuyoAist works on Windows, Linux and Mac OS X. It uses the Gearbox
``flexiport'' library for communications with the laser. This must be
installed. A compiled copy of this library is included in the installer for
users of Windows.

\section{Installation}
\label{sec:installation}

\subsection{Binary}

Users of Windows can install the component using the binary installer. This
will install the component and all its necessary dependencies. It is the
recommended method of installation in Windows.

\begin{enumerate}
  \item Download the installer from the website.
  \item Double-click the executable file to begin installation.
  \item Follow the instructions to install the component.
  \item You may need to restart your computer for environment variable changes
  to take effect before using the component.
\end{enumerate}

The component can be launched by double-clicking the
\verb|rtc_hokuyoaist_standalone| executable. The \verb|rtc_hokuyoaist| library
is available for loading into a manager, using the initialisation function
\verb|rtc_init|.

\subsection{From source}

Follow these steps to install RTC:HokuyoAist from source in any operating
system:

\begin{enumerate}
  \item Download the source, either from the repository or a source archive,
  and extract it somewhere.

  \verb|tar -xvzf rtc_hokuyoaist-2.0.0.tar.gz|
  \item Change to the directory containing the extracted source.

  \verb|cd rtc_hokuyoaist-2.0.0|
  \item Create a directory called ``build'':

  \verb|mkdir build|
  \item Change to that directory.

  \verb|cd build|
  \item Run cmake or cmake-gui.

  \verb|cmake ../|
  \item If no errors occurred, run make.

  \verb|make|
  \item Finally, install the component. Ensure the necessary permissions to
  install into the chosen prefix are available.

  \verb|make install|
  \item The install destination can be changed by executing ccmake and changing
  the variable \verb|CMAKE_INSTALL_PREFIX|.

  \verb|ccmake ../|
\end{enumerate}

The component is now ready for use. See the next section for instructions on
configuring the component.

RTC:HokuyoAist can be launched in stand-alone mode by executing the
\verb|rtc_standalone| executable (installed into \verb|${prefix}/bin|).
Alternatively, \verb|librtc.so| can be loaded into a manager, using the
initialisation function \verb|rtc_init|. This shared object can be found in
\verb|${prefix}/lib| or \verb|${prefix}/lib64|.


\section{Configuration}
\label{sec:configuration}

The available configuration parameters are described in
Table~\ref{tab:config_params}.

\begin{table}[t]
  \centering
  \begin{tabularx}{\columnwidth}{lX}
    \toprule
    Parameter & Effect \\
    \midrule
    port\_opts & Change the options used to open the port to the laser. See flexiport for details. \\
    start\_angle & The angle to begin scanning at, in radians. Set this to 0 to begin at the default for a full scan. \\
    end\_angle & The angle to stop scanning at, in radians. Set this to 0 to end at the default for a full scan. \\
    cluster\_count & The number of readings to group into each cluster. The default is 1. \\
    enable\_intensity & When set to true, intensity data will also be sent. \\
    high\_sensitivity & Some models feature a high sensitivity mode. Use this to enable it. \\
    pull\_mode & Switch to pull mode. You will need to request each scan through the HokuyoAist service interface. \\
    new\_data\_mode & When set to true, new data will be requested each time. Otherwise, the most recent data will be sent. \\
    error\_time & The minimum time between errors, in seconds. If errors occur closer together than this value, the component will transition to the error state. Otherwise, it will attempt to reset the laser and continue. \\
    x, y, z & The laser's position in 3D space. \\
    roll, pitch, yaw & The laser's orientation in 3D space. \\
    \bottomrule
  \end{tabularx}
  \caption{Available configuration parameters.}
  \label{tab:config_params}
\end{table}

\section{Ports}
\label{sec:port}

The ports provided by the component are described in Table~\ref{tab:ports}.

The communications protocol used by the Hokuyo lasers reports error values
within the range data using values less than 20. This component filters out
these values and replaces them with 0m. You should consider all zero values as
unknown when processing sensor data.

\begin{table}[t]
  \centering
  \begin{tabularx}{\columnwidth}{lllX}
    \toprule
    Name & Type & Data type & Purpose \\
    \midrule
    ranges & OutPort & RTC::RangeData & Range data scanned by the laser. \\
    intensities & OutPort & RTC::IntensityData & Intensity data scanned by the laser. Only available when intensity data is enabled. \\
    ranger & Service & Ranger & Service port providing the generic RTC::Ranger interface. \\
    & & HokuyoAist & Service port providing the HokuyoAist interface, for specialised features not supported by the generic RTC::Ranger interface. \\
    \bottomrule
  \end{tabularx}
  \caption{Available ports.}
  \label{tab:ports}
\end{table}

\section{Examples}
\label{sec:examples}

An example configuration file is provided in the
\verb|${prefix}/share/rtc_hokuyoaist/examples/conf/| directory.

\section{Changelog}

\subsection{2.0}

\begin{itemize}
  \item Support hokuyo\_aist library v2 API.
\end{itemize}

\end{document}
